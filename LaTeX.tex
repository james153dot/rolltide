\documentclass{article}
\usepackage{graphicx}
\usepackage{unicode-math}
\usepackage{caption}
\usepackage{hyperref}
\usepackage{atbegshi}% http://ctan.org/pkg/atbegshi
\usepackage[a4paper,
            bindingoffset=0.2in,
            left=1in,
            right=1in,
            top=1in,
            bottom=1in,
            footskip=.25in]{geometry}
\hypersetup{
    colorlinks=true,
    linkcolor=blue,
    filecolor=magenta,      
    urlcolor=cyan,
    }
\captionsetup[figure]{font=small}
\setlength\parindent{0pt}
\AtBeginDocument{\AtBeginShipoutNext{\AtBeginShipoutDiscard}}
\title{Equations Found in our Code}
\author{James Lu, Denis Aslangil}
\date{Summer 2024}
\begin{document}
\begin{center}
\maketitle
\section*{Change of Coordinate System}
\setcounter{page}{1}
\begin{Large}
The Cartesian system uses \(x, y, z\), and is most commonly used.\\
We use spherical variables \(r, \theta, \phi\) in the Spherical system. 
Converting from Cartesian to Spherical requires use of these equations.
\bigbreak
\textbf{Cartesian to Spherical system conversion}

\begin{small} Typically both $\theta$ and $\phi$ come in degrees, but conversion is simple. \end{small}
\[
x = r \cos(\theta) \sin(\phi) 
\]
\[
y = r \sin(\theta) \sin(\phi)
\]
\[
z = r \cos(\phi)
\]
\textbf{Spherical to Cartesian system conversion}

\begin{small}
These conversions are used most in lines 116-120 of our Sphere3D code.
\end{small}
\[
r = \sqrt{(x-x_c)^2 + (y-y_c)^2 + (z-z_c)^2}
\]
\[
\theta = \arctan \left( \frac{y-y_c}{x-x_c} \right)
\]
\[
\phi = \arccos \left( \frac{(z-z_c)}{\sqrt{(x-x_c)^2 + (y-y_c)^2 + (z-z_c)^2}} \right)
\]
\end{Large}


\begin{figure}[h!]
    \centering
    \includegraphics[width=0.4\linewidth]{image.png}
    \caption{https://byjus.com/maths/spherical-coordinates/}
    \label{fig:enter-label}
\end{figure}
\pagebreak
\begin{Large}
\section*{New Velocity Terms}
Our velocity terms come from Stanford AA200 - Applied Aerodynamics.
It is in the Cartesian system and defines uniform flow past a sphere.
\bigbreak
\textbf{Direct terms from textbook}

\begin{small}\href{https://web.stanford.edu/~cantwell/AA200_Course_Material/AA200_Lectures/AA200_Ch_10_Elements_of_potential_flow_Vortex_Sticks_and_Stokes_Flow_Brian_Cantwell.pdf}{Equation 10.75}\end{small}
\[
U_x = U_\infty (1-\frac{3(R_{sphere})^3x^2}{2r^5} + \frac{(R_{sphere})^3}{2r^3} )
\]
\[
U_y = -U_\infty \frac{3(R_{sphere})^3xy}{2r^5} )
\]
\[
U_z = -U_\infty \frac{3(R_{sphere})^3xz}{2r^5} 
\]
\bigbreak
\textbf{For our specific case}
\bigbreak
We used diameter as opposed to radius in our declaration of variables, giving a mathematically identical equation but one that looks slightly different.
\bigbreak
\[
U_x = U_\infty (1-\frac{3(D_{sphere})^3x^2}{16r^5} + \frac{(D_{sphere})^3}{16r^3} )
\]
\[
U_y = -U_\infty \frac{3(D_{sphere})^3xy}{16r^5} )
\]
\[
U_z = -U_\infty \frac{3(D_{sphere})^3xz}{16r^5}
\]
\bigbreak
These equations are declared, defined, and used in lines 193-195 of
file UniformFlow3D.

\end{Large}
\end{center}
\end{document}
